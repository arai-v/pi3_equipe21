O desenvolvimento do \textbf{TPG System} ao longo do segundo semestre de 2024 foi estruturado com o uso de diversas metodologias e ferramentas, que garantiram a eficiência do projeto. A metodologia Scrum foi adotada para identificar as necessidades reais dos usuários e garantir que o desenvolvimento das competências no componente Arte para os estudantes do Ensino Fundamental fosse bem-sucedido. Durante o processo, as necessidades dos usuários foram coletadas por meio de análise de dados de desempenho, definindo as funcionalidades principais do sistema. A equipe se organizou em ciclos de trabalho conhecidos como sprints, onde tarefas específicas foram definidas, priorizadas e executadas em um período limitado. Ao final de cada sprint, a equipe fez uma revisão, adaptando o planejamento de acordo com o feedback obtido.

A metodologia \textbf{Kanban} foi utilizada para organizar as etapas do projeto e priorizar as ações ao longo do semestre. Este sistema visual ajudou a registrar e gerenciar as tarefas do projeto, utilizando cartões e quadros para mapear o progresso. As etapas foram divididas em tópicos importantes, como a análise da devolutiva do artigo, reescrita de documentos e o acompanhamento das entregas para avaliação final, com cada tarefa sendo movida para a próxima fase à medida que avançava.

No início do desenvolvimento, o uso do \textbf{Figma} foi essencial para o design do \textbf{Módulo Gestor}, a plataforma que será utilizada pelos professores. A equipe de design trabalhou na criação de protótipos interativos, que foram testados para validar a usabilidade e a fluidez da interface. Após essa validação, a nova identidade visual foi aplicada e as telas foram refinadas, garantindo que as informações fossem organizadas de maneira clara e acessível.

Para o design dos personagens do jogo, o processo criativo foi essencial. A Hero Forge foi utilizada para modelar os personagens em 3D, criando uma base que refletisse as características desejadas. Em seguida, as imagens dos personagens foram refinadas usando o \textbf{Inkscape}, para garantir que as telas do jogo fossem estruturadas de forma intuitiva e visualmente atrativa. O Krita e o Photoshop foram utilizados para vetorização do personagem (desenhando digitalmente), pintura digital e a animação das sprites, criando sequências de movimento para os personagens, que foram cuidadosamente testadas para garantir fluidez e realismo.

Durante o desenvolvimento do jogo, as plataformas de \textbf{IA Generativa}, como \textbf{Canva}, \textbf{Leonardo IA} e \textbf{Seeart}, foram usadas para criar imagens conceituais e enriquecem o visual do jogo com gráficos inovadores baseados em prompts e imagens de referência. As imagens geradas por \textbf{IA} ajudaram a acelerar o processo criativo, fornecendo variações e novas ideias para elementos visuais do jogo.

O \textbf{MongoDB}, como banco de dados não relacional foi escolhido pela sua flexibilidade e escalabilidade. A equipe organizou o modelo de dados para suportar tanto o armazenamento das informações dos usuários quanto o acompanhamento do progresso dos estudantes no jogo. A Interface de Programação de Aplicações (API) em Node.js foi desenvolvida para conectar o banco de dados com o front-end (tela de interação do usuário junto ao sistema) do jogo e o \textbf{Módulo Gestor}, permitindo a atualização de perfis de usuários e o acompanhamento da evolução dos estudantes durante o jogo. Essa API garantiu a comunicação eficiente entre as diferentes plataformas, permitindo que as informações fossem sincronizadas em tempo real.

Para o desenvolvimento do sistema, o Visual Studio Code (VS Code) foi o ambiente de desenvolvimento integrado \textbf{(IDE)} escolhido, pois oferece uma interface simples e eficiente para codificação. A linguagem de programação Python foi usada como a principal linguagem para a codificação, sendo escolhida pela sua versatilidade e robustez. As bibliotecas \textbf{Tkinter} e \textbf{Pygame} foram utilizadas para criar as interfaces gráficas e para o desenvolvimento do jogo. \textbf{Tkinter} foi empregado para o \textbf{Módulo Gestor}, permitindo a criação de interfaces intuitivas para os professores. Já o \textbf{Pygame} foi fundamental para o desenvolvimento do jogo, garantindo uma experiência imersiva para os estudantes, com controles interativos, gráficos e animações.
